%resume.tex
% vim:set ft=tex spell:

\documentclass[10pt]{article}
\usepackage{palatino}
%\usepackage{fullpage}
\usepackage{amsmath}
\usepackage{amssymb}
\usepackage{url}
\usepackage{textcomp}
\usepackage{mdwlist}
\usepackage[usenames]{color}
\usepackage[letterpaper, portrait, margin=1in, top=0.5in, bottom=0.5in]{geometry}
%\topmargin=-1.0in
%\bottom=1.0in
\raggedright

\pagestyle{empty}

% indentsection style, used for sections that aren't already in lists
% that need indentation to the level of all text in the document
\newenvironment{indentsection}[1]%
{\begin{list}{}%
	{\setlength{\leftmargin}{#1}}%
	\item[]%
}
{\end{list}}

% opposite of above; bump a section back toward the left margin
\newenvironment{unindentsection}[1]%
{\begin{list}{}%
	{\setlength{\leftmargin}{-0.5#1}}%
	\item[]%
}
{\end{list}}

\newenvironment{changemargin}[2]{%
  \begin{list}{}{%
    \setlength{\topsep}{0pt}%
    \setlength{\leftmargin}{#1}%
    \setlength{\rightmargin}{#2}%
    \setlength{\listparindent}{\parindent}%
    \setlength{\itemindent}{\parindent}%
    \setlength{\parsep}{\parskip}%
  }%
  \item[]}{\end{list}
}

\newcommand{\lineover}{
	\begin{changemargin}{-0.05in}{-0.05in}
		\vspace*{-8pt}
		\hrulefill \\
		\vspace*{-2pt}
	\end{changemargin}
}

\newenvironment{body} {
	\begin{changemargin}{-0.25in}{-0.5in}
  }
	{\end{changemargin}
}

\newenvironment{fullbody} {
	\begin{changemargin}{-0.5in}{-0.5in}
  }
	{\end{changemargin}
}

% format two pieces of text, one left aligned and one right aligned
\newcommand{\headerrule}[1]{
    \vspace*{5pt}
    \begin{changemargin}{-0.5in}{-0.5in}
        \scshape{#1}\\
    \lineover
    \end{changemargin}
}

% format two pieces of text, one left aligned and one right aligned
\newcommand{\headerrow}[2]
{\begin{tabular*}{\linewidth}{@{}l@{\extracolsep{\fill}}r}
	#1 &
	#2 \\
\end{tabular*}}

% make "C++" look pretty when used in text by touching up the plus signs
\newcommand{\CPP}
{C\nolinebreak[4]\hspace{-.05em}\raisebox{.22ex}{\footnotesize\bf ++ }}

\begin{document}
\thispagestyle{empty}
\begin{center}
{\LARGE \scshape{Brian H. Hulette}}
\end{center}

\vspace{+1.0em}
% TODO: Get this on to one line
\begin{fullbody}
\headerrow
{4000 City Walk Way Apt. 337}
{919.943.5524}

\headerrow
{Charlottesville, VA 22902}
{hulettbh@gmail.com}

\headerrow
{} %]{\textbf{Security Clearance:} Active TS/SCI with CI Polygraph}
{\url{http://theneuralbit.com}}
\end{fullbody}

\headerrule{Education}
\begin{body}
    \headerrow
        {\textbf{M.Eng Electrical Engineering (DSP/Communications) - Virginia Tech}}
        {\textbf{May 2015}}
    \\
    \headerrow
        {\emph{Falls Church, VA - National Capital Region Campus}}
        {\emph{4.0 GPA}}
    \vspace{-1.7em}
    % TODO: Figure out how to make this indented correctly

    \begin{description*}
        \item[Coursework:]
        Advanced Digital Communications, Detection and Estimation Theory, \\
        Introduction to Algorithms, Introduction to Network Security, Radar System Design, \\
        Network Architectures and Protocols
        \item[Project:]
            \emph{Simulation of Various Channelizer Structures
                  Directed by a Cyclostationary Detector}
    \end{description*}

    \vspace{+0.5em}

    \headerrow
        {\textbf{B.S. Computer Engineering - Rose-Hulman Institute of Technology}}
        {\textbf{May 2011}}
    \\
    \headerrow
        {\emph{Terre Haute, IN}}
        {\emph{Summa cum laude}}
    \vspace{-1.7em}
    % TODO: Figure out how to make this indented correctly
    \begin{description*}
        \item[Coursework:]
        Digital Signal Processing, DSP System Design, Communication Systems, \\
        Electronic Music Synthesis
    \end{description*}
\end{body}

\headerrule{Experience}
\begin{body}
    \vspace{+0.5em}
    \headerrow
        {\textbf{Commonwealth Computer Research, inc. (CCRi)}}
        {\textbf{April 2016 - Present}}
    \headerrow
        {\emph{Data Scientiest}}
        {\emph{Charlottesville, VA}}
    \begin{itemize*}
        \item Designed and implemented a novel caching approach to accelerate
              rendering millions of geo-spatial observations in a web browser
              interface
        \item Developed a new interactive time control for the same interface
        \item Currently using the emscripten compiler to port a C++/Qt
              Application to Javascript

    \end{itemize*}

    \vspace{+0.5em}
    \headerrow
        {\textbf{n\texttildelow{}ask, inc. Signal Processing Systems}}
        {\textbf{July 2011 - March 2016}}
    \headerrow
        {\emph{Associate}}
        {\emph{Fairfax, VA}}
    \begin{itemize*}
        % TODO: Limit margins of this list so it stop just under "July"
        %\item Created innumerable python and \CPP tools for extracting and
        %      analyzing signal data.
        %\item Worked on tools for porting Software-Defined Radio (SDR)
        %      components into various frameworks.
        \item Developed and deployed a system for automatic signal
              detection and classification.
        %      Detection information is stored
        %      in a SQL database which can be viewed by analysts using a custom
        %      Qt interface or a webpage.
        \item Contributed to an innovative set of Qt tools for viewing multiple
              sets of spectral data and identify common features between them
        \item Developed a WebSocket interface for the next major
              version of X-Midas (M50), an SDR framework used by the Intelligence
              Community, to bring signal analysis to the web.
        \item Efficiently implemented various DSP algorithms, including
              a Maximum Likelihood Sequence Estimator (MLSE) demod and a wideband
              detector for identifying signals with repeated sync patterns.
        \item Implemented two techniques for identifying corresponding
              uplink/downlink frequencies for a particular signal. One using
              efficient DSP algorithms and the other using analytic tools
              leveraging MongoDB.
    \end{itemize*}

    %\vspace{+0.5em}
    %\headerrow
    %    {\textbf{Rose-Hulman Senior Project - Wireless Video Viewing Device}}
    %    {\textbf{Aug 2010 - May 2011}}
    %\headerrow
    %    {\emph{Student}}
    %    {\emph{Terre Haute, IN}}
    %\begin{itemize*}
    %    \item Created an embedded transmitter receiver pair to transfer low
    %          bit-rate video
    %    \item Configured two TI DaVinci video processors to encode/decode an
    %          H.264 video stream and transfer it via our client's low bit-rate
    %          radios
    %    \item Made Kernel modifications and wrote \CPP applications for data
    %          capture, transfer, and display
    %\end{itemize*}

    \vspace{+0.5em}
    \headerrow
        {\textbf{Duke University Center for In Vivo Microscopy}}
        {\textbf{Summer 2010}}
    \headerrow
        {\emph{Undergraduate Research Assistant}}
        {\emph{Durham, NC}}
    \begin{itemize*}
        \item Implemented a 3D spherical Hough Transform in MATLAB to identify
              and measure glomeruli in extremely high resolution (15
              $\mu$m voxels) MRI images of a kidney
        \item Created visualizations of 3D brain, kidney and heart MRI data
    \end{itemize*}
    %\vspace{+0.5em}
    %\headerrow
    %    {\textbf{Rose-Hulman Ventures - Vision Project}}
    %    {\textbf{Nov 2009 - May 2010}}
    %\headerrow
    %    {\emph{Intern}}
    %    {\emph{Terre Haute, IN}}
    %\begin{itemize*}
    %    \item Created a system to automatically calibrate a stereo vision system
    %          with a robotic arm's coordinate frame
    %    \item Developed software to control a Motoman HP-6 robotic arm as well
    %          as MATLAB calibration algorithms
    %\end{itemize*}
\end{body}
\headerrule{Technical Skills}
\begin{body}
\begin{description*}
	\item[Areas of Interest:]
	Digital Signal Processing, Machine Learning, Algorithm Development
	\item[Languages:]
	Python, C/\CPP, Javascript, MATLAB, \LaTeX
	\item[Other:]
    git, Mercurial, Linux, X-Midas, Windows
\end{description*}
\end{body}

\headerrule{Publications}
\begin{body}
B. Hulette and A. Zaghloul. ``Simulation of Various Channelizer Structures
Directed by Cyclostationary Detector," presented at the 1\textsuperscript{st}
URSI Atlantic Radio Science Conference, Gran Canaria, Canary Islands, 2015.

\vspace{0.25cm}

L. Xie, R. Cianciol, B. Hulette, H. Won Lee, Y. Qi, G. Cofer, G.A. Johnson.
"Magnetic resonance histology of age-related nephropathy in the Sprague
Dawley rat." \emph{Toxicologic Pathology}, vol. 40 no. 5, pp. 764-778, July 2012.
\end{body}


\end{document}
